\documentclass{article}
\usepackage{amsmath, amssymb}

\begin{document}

\section*{ダイクストラのアルゴリズム:帰納法による正しさの証明}

ダイクストラのアルゴリズムの正しさを帰納法を用いて証明する。  
ここで、グラフ $G$ は入力グラフ、$s$ は始点、$\ell(uv)$ は辺 $u$ から $v$ への長さ、$V$ は頂点集合とする。

\subsection*{アルゴリズム(Dijkstra(G, s))}

\begin{itemize}
  \item $\forall u \in V \setminus \{s\}$ に対して $d(u) \gets \infty$
  \item $d(s) \gets 0$
  \item $R \gets \emptyset$
  \item \textbf{while} $R \neq V$:
  \begin{itemize}
    \item $d(u)$ が最小の $u \notin R$ を選ぶ
    \item $R \gets R \cup \{u\}$
    \item $u$ に隣接するすべての $v$ に対して:
    \begin{itemize}
      \item \textbf{if} $d(v) > d(u) + \ell(u, v)$:
      \item \quad $d(v) \gets d(u) + \ell(u, v)$
    \end{itemize}
  \end{itemize}
\end{itemize}

アルゴリズムによるラベルを $d(v)$、始点から $v$ への最短距離を $\delta(v)$ とする。  
すべての $v$ に対して $d(v) = \delta(v)$ を示す。

\subsection*{補題:$\forall x \in R$, $d(x) = \delta(x)$}

\textbf{帰納法による証明:}

\textbf{基底ケース ($|R| = 1$):}  
このとき $R = \{s\}$ なので $d(s) = 0 = \delta(s)$ は明らかに正しい。

\textbf{帰納法の仮定:}  
$u$ を最後に $R$ に追加された頂点とし、$R' = R \cup \{u\}$ とする。  
仮定:すべての $x \in R'$ に対して $d(x) = \delta(x)$。

\textbf{帰納ステップ:}  
仮に $s$ から $u$ への最短経路を $Q$、その長さを $\ell(Q)$ とし、$\ell(Q) < d(u)$ と仮定する。  
$Q$ は $R'$ 内から出発し、途中で $R'$ を出る必要がある。  
最初に $R'$ を抜ける辺を $xy$、$Q$ の $s$ から $x$ までの部分経路を $Q_x$ とする。

\[
\ell(Q_x) + \ell(xy) \leq \ell(Q)
\]

帰納仮定より $d(x) \leq \ell(Q_x)$

\[
d(x) + \ell(xy) \leq \ell(Q)
\]

また $y$ は $x$ に隣接しており、更新されているので:

\[
d(y) \leq d(x) + \ell(xy)
\]

$u$ は最小の距離ラベルを持つため:

\[
d(u) \leq d(y)
\]

これらをまとめると:

\[
d(u) \leq d(y) \leq d(x) + \ell(xy) \leq \ell(Q) < d(u)
\]

矛盾が生じたため、そんな経路 $Q$ は存在しない。  
よって $d(u) = \delta(u)$ が示された。

\textbf{以上により、$R = V$ であればアルゴリズムは正しく動作する。}

\end{document}
